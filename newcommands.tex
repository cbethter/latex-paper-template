%\newcommand\CC{\Lang{\mbox{R}}\xspace}
%\newcommand\Lang[1]{\textsc{#1}}
%\newcommand{\kw}[1]{\texttt{\textbf{#1}}}
%\newcommand{\cd}[1]{\texttt{#1}}
\newcommand{\HRule}{\rule{\linewidth}{0.2mm}\\}
%
\newcommand\Exp{\ensuremath{\mathbb{E}}\xspace}
\newcommand\Var{\ensuremath{\mathbb{V}}\xspace}

\renewcommand\Pr{\ensuremath{\text{Pr}}\xspace}

\newcommand\F{\ensuremath{\mathbb{F}}\xspace}

\newcommand\N{\ensuremath{\mathbb{N}}\xspace}
\newcommand\Z{\ensuremath{\mathbb{Z}}\xspace}
\newcommand\Q{\ensuremath{\mathbb{Q}}\xspace}
\newcommand\R{\ensuremath{\mathbb{R}}\xspace}
\newcommand\C{\ensuremath{\mathbb{C}}\xspace}
\newcommand\I{\ensuremath{\mathbb{I}}\xspace}
%
\newcommand\norm[1]{\ensuremath{\lVert#1\rVert}}
\newcommand\abs[1]{\ensuremath{\lvert#1\rvert}}
\newcommand\ceil[1]{\ensuremath{\lceil#1\rceil}}
\newcommand\floor[1]{\ensuremath{\lfloor#1\rfloor}}
\newcommand\set[1]{\ensuremath{\{#1\}}}
\newcommand\angular[1]{\ensuremath{\langle#1\rangle}}
%
\newcommand\Norm[1]{\ensuremath{\left\lVert#1\right\rVert}}
\newcommand\Abs[1]{\ensuremath{\left\lvert#1\right\rvert}}
\newcommand\Ceil[1]{\ensuremath{\left\lceil#1\right\rceil}}
\newcommand\Floor[1]{\ensuremath{\left\lfloor#1\right\rfloor}}
\newcommand\Set[1]{\ensuremath{\left\{#1\right\}}}
\newcommand\Angular[1]{\ensuremath{\left\langle#1\right\rangle}}
%
\newcommand\Tr{\ensuremath{\text{Tr}}}
%\newcommand\dom{\ensuremath{\mathbf{dom}}}
\newcommand\dom[1]{\unskip\ensuremath{\,\mathbf{dom}(#1)}}
%
%\DeclareMathOperator{\dom}{\mathbf{dom}}
%
\newcommand\laplacian{\ensuremath{\mathcal{L}}}
\newcommand\SO[1]{\ensuremath{\mathbf{SO}(#1)}}
\newcommand\var[1]{\ensuremath{\mathbf{var}(#1)}}
\newcommand\EIG[1]{\ensuremath{\mathbf{EIG}(#1)}}
\newcommand\0{\ensuremath{\mathbf{0}}}
\newcommand\1{\ensuremath{\mathbf{1}}}
\newcommand\B{\ensuremath{\mathbf{B}}}
\newcommand\U{\ensuremath{\mathbf{U}}}
\newcommand\nextt{\ensuremath{\mathrm{next}}}
%
%
%% Define a label for every chapter and section, which has the same name as the entity
%\newcommand{\Chapter}[1]{\chapter{#1} \label{ch:#1}}
\newcommand{\Section}[1]{\section{#1} \label{sec:#1}}
\newcommand{\Subsection}[1]{\subsection{#1} \label{ssec:#1}}
%
%
%\newcommand{\LOOM}{\ensuremath{\cal{LOOM}}\xspace}
%\newcommand{\PolyTOIL}{\textbf{PolyTOIL}\xspace}
%
\newtheorem{theorem}{Theorem}[section]
\newtheorem{definition}[theorem]{Definition}
\newtheorem{lemma}[theorem]{Lemma}
\newtheorem{corollary}[theorem]{Corollary}
\newtheorem{fact}[theorem]{Fact}
\newtheorem{observation}[theorem]{Observation}
\newtheorem{example}[theorem]{Example}
%% theorems without counter
\newtheorem*{notation}{Notation}
\newtheorem*{notation*}{Notation}
%
\renewcommand{\algorithmicrequire}{\textbf{Input:}}
\renewcommand{\algorithmicensure}{\textbf{Output:}}

%
%\newcommand\Cls[1]{\textsf{#1}}
%\newcommand\Fig[1]{Figure~\ref{Figure:#1}}
%
%
%%% Algorithm %%%
\newrefformat{alg}{Algorithm~(\ref{#1})}%\glqq\titleref{#1}\grqq \ auf Seite \pageref{#1}}
%%% Equations %%%
\newrefformat{eq}{Equation~(\ref{#1})}%\glqq\titleref{#1}\grqq \ auf Seite \pageref{#1}}
%%% Problem %%%
\newrefformat{pr}{Problem~(\ref{#1})}%\glqq\titleref{#1}\grqq \ auf Seite \pageref{#1}}
%%% Definition %%%
\newrefformat{def}{Definition~\ref{#1}}%\glqq\titleref{#1}\grqq \ auf Seite \pageref{#1}}
%%% Figures %%%
\newrefformat{fig}{Figure~(\ref{#1})} %\glqq\titleref{#1}\grqq \ auf Seite \pageref{#1}}
%%% Tables %%%
\newrefformat{tab}{Table~(\ref{#1})} %\glqq\titleref{#1}\grqq \ auf Seite \pageref{#1}}
%
\newrefformat{th}{Theorem~(\ref{#1})} %\glqq\titleref{#1}\grqq \ auf Seite \pageref{#1}}
%
\newrefformat{ch}{Chapter~\ref{#1}} %\glqq\titleref{#1}\grqq \ auf Seite \pageref{#1}}
%
\newrefformat{sec}{Section~\ref{#1}} %\glqq\titleref{#1}\grqq \ auf Seite \pageref{#1}}
%
\newrefformat{ssec}{Subsection~\ref{#1}} %\glqq\titleref{#1}\grqq \ auf Seite \pageref{#1}}
%
\newcommand\parenref[1]{(\ref{#1})}
%
%\def\listofnotations{\input{notation} \clearpage}
%\def\addnotation #1: #2#3{$#1$\> \parbox{5in}{#2 \fill  \pageref{#3}}\\}
%\def\newnot#1{\label{#1}}
%
%\usepackage[refpage]{nomencl}
\usepackage{nomencl}
\makenomenclature
%\renewcommand{\nomname}{List of Notations}
%\renewcommand*{\pagedeclaration}[1]{\unskip\dotfill\hyperpage{#1}}
%\nomlabelwidth=25mm
%
%\usepackage{makeidx}
%\makeindex
%
\usepackage{comment}
%\includecomment{comment}
%
%\newenvironment{excerpt}{\begin{quote}\begin{minipage}\textwidth}{\end{minipage}\end{quote}}
%
%\renewcommand{\textfraction}{0.01}
%
